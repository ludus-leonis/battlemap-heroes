\newcommand{\bmhMissionIIIHeadline}{A3: Das Banditenlager}
\newcommand{\bmhMissionIIIToc}{A3: Das Banditenlager}
\newcommand{\bmhMissionIII}{\fsNormalB\selectfont

	\parchment{
		{\fffancy\noindent
			Die Handelsstraße ist unsicher geworden. Banditen überfallen die Karawanen, sogar einen Zirkus hat es neulich erwischt. Die Händlergilde hat euch eine Prämie von 10\gold\ pro Goblin und 20\gold\ pro Ork versprochen, die angeblich hinter den Überfällen stecken, wenn ihr ihre Waren zurückbringt. In Fässern versteckt habt ihr euch stehlen lassen, damit euch die Banditen direkt in ihre, bisher unauffindbare Höhle bringen.
		}

		\medskip

		\noindent Die Helden starten bei den Fässern. So nicht anders angegeben, sind die Türen bloß muffige Vorhänge.

		\medskip

		\dropping[2]{1} \say{Ihr spring aus den Fässern und findet euch in einer natürlichen Höhle wieder. Fackeln an den Wänden sorgen für düsteres Licht. Ihr könnt Kisten ausmachen -- wohl die Beute der Bande. Goblins sind dabei, die Güter zu sichten. Ein Vorhang verbirgt den Ausgang.}

		\search~Die Kisten enthalten die Besitztümer der Händergilde.

		\medskip

		\dropping[2]{2} \say{In dieser großen Höhle steht der Wagen, auf dem eure Fässer transportiert wurden. Er wird gerade von einem Goblin abgesucht. Weitere Goblins und ein Ork stehen um ein Lagerfeuer und verspeisen das Pferd. Vom Fahrer ist nichts zu sehen. Ein Durchgang im Süden führt ins Freie.}

		Von den zwei südlichen Feldern kann die Mission verlassen werden.

		\medskip

		\dropping[2]{3} \say{Enge, natürlich geformte Gänge führen in den Fels. Die Luft scheint hier zu stehen. Es ist stockdunkel.}

		\trap~Stolperseil (GE\TN1): Die Vorhänge zur Hundehöhle öffnen sich. Diese wird aufgedeckt, wenn sie es noch nicht ist.
	}

	\columnbreak

	\vspace*{1cm}

	\newpage

	\vspace*{1cm}

	\columnbreak

	\parchment{
		\dropping[2]{4} \say{In dieser kleinen Schlafhöhle stehen zwei schmutzige Betten. Niemand ist zu sehen.}

		In diesen Höhlen ist es dunkel.

		\search~Unter den Betten finden sich 10\gold.

		\medskip

		\dropping[2]{5} \say{Tageslicht fällt durch Risse in der Decke in diese Höhle. Drei Wachhunde jagen einem Schmetterling hinterher, als sie euch bemerken~\ldots}

		Benutze Wölfe als Hunde.

		\medskip

		\dropping[2]{6} \say{Hier wurde von einer vermutlich etwas größeren Höhle der hintere Teil mittels einer Holzwand aus einfachen Planken abgetrennt. In ihr befinden sich vier Holztüren, die alle gleich aussehen.}

		Hinter den Türen befinden sich kleine Holzkammern. In drei davon lauern ein Löwe, ein Puma und ein Tiger -- die dem überfallenen Zirkus gehörten. Die Tiere sind hungrig und greifen sofort an, wenn ihre Türen geöffnet werden.

		\search~In der Nordwand befindet sich eine Geheimtüre.

		\medskip

		\dropping[2]{7} \say{Ihr habt die Anführer der Bande gefunden: Orks brüten über einem Buch, das auf einem Tisch liegt. In einer Ecke des Raumes steht ein Bett.}

		Der nördliche Ork besitzt einen Schild+1, seine Werte sind entsprechend anzupassen.

		\search~Hier findet sich das Beuteregister der Banditen. Es scheint, als stünde ein Mitglied der Händergilde selbst hinter den Überfällen. Die Gilde zahlt 50\gold\ für dieses Beweismittel.

		\medskip

		\noindent
		Missionstufe: 1 \\
		Streunendes Monster: Goblin
	}
}
