\newcommand{\bmhMissionIHeadline}{A1: Die Krypta der Helden}
\newcommand{\bmhMissionIToc}{A1: Die Krypta der Helden}
\newcommand{\bmhMissionI}{\fsNormalB\selectfont

	\parchment{
		{\fffancy\noindent
			Die Priester des örtlichen Tempels haben euch um Hilfe gebeten. Ein Adept ist von einem Skelett angefallen worden, als er in der Krypta früherer Helden eine Reliquie reinigte. Er konnte fliehen und das Gemäuer verschließen. Ihr sollt hinabsteigen und eure Vorgänger befrieden.
		}

		\bigskip

		\noindent Die Helden starten auf der Treppe.

		\bigskip

		\dropping[2]{1} \say{Eine breite Marmortreppe führt in einen Rundgang, in dessen Wänden sich zugemauerte Grabnischen befinden. Eine steinerne Doppeltüre mit der Aufschrift \say{Hier ruhen unsere Helden} führt nach Norden. In die Wände eingelassene Sonnensteine erhellen die sehr saubere Anlage. Aus einigen, von innen aufgebrochen Nischen sind Skelette geklettert. Als sie euch bemerken, zeigen sie auf euch und greifen an!}

		\search~In angrenzenden Nischen liegen Grabbeigaben im Wert von je 5\gold.

		\bigskip

		\dropping[2]{2} \say{Auch die Wände dieses gut gepflegten Ganges sind voller Grabnischen. In der Nordwand befindet sich ein schmales Becken. Wasser fließt aus dem geöffneten Mund einer großen Steinfratze und füllt es. Weitere Skelette klettern gerade aus ihren Nischen.}

		\search~Wird hier gesucht, spricht die Fratze: \say{Gebt mir euer Gold und ich werde euch heilen.} Benutzt ein Held das Becken, heilt er 1LP pro 1G, das er hinein wirft. Wer Gold entnimmt, verliert entsprechend viele LP. Wird die Fratze zur Krypta befragt, sagt sie: \say{Das Übel haust zu meiner Rechten. Zu meiner Linken werdet ihr etwas nützliches finden.}
	}

	\columnbreak

	\vspace*{1cm}

	\newpage

	\vspace*{1cm}

	\columnbreak

	\parchment{
		\dropping[2]{3} \say{Dieser Raum enthält keine Gräber, sondern einen Altar. Darauf liegt ein silberner Streitkolben. Zwei Statuen, steinerne Ritter, stehen in den Ecken.}

		\trap~Stelle keine Figuren für die zwei Statuen (*) auf den Plan, sondern zeichne sie nur auf der Karte ein. Sollte ein Held eines der Fallen-Felder betreten, werden diese steinernen Wächter lebendig und greifen an. %(IN\TN2)

		\search~Der geweihte Streitkolben+1 (ST vs. AB) hilft im Kampf gegen Geister. Wer damit diese bekämpft, darf einen zusätzlichen Würfel würfeln. Das heilige Relikt dürfen die Helden nach einer erfolgreichen Mission behalten.

		\bigskip

		\dropping[2]{4} \say{In dieser Grabkammer stehen vier steinerne Sarkophage. Der nordwestliche ist durch eine umgestürzte Wand beschädigt worden, und hat einen Helden vergangener Tage in seiner Ruhe gestört. Dieser ist erzürnt als Wiedergänger auferstanden und hat wohl seine ehemaligen Kumpanen als Zombies und seine Gefolgsleute als Skelette erweckt. In der Nordwand klafft ein Loch.}

		\bigskip

		\dropping[2]{5} \say{Ein Guhl hat von einem nahen Brunnenschacht aus diesen Gang gegraben, ehe er von der umstürzenden Wand erschlagen wurde. Seine Überreste sind im Schutt noch gut auszumachen.}

		\search~Wer den Guhl durchsucht, findet ein bronzenes Medaillon im Wert von 10\gold.

		\bigskip

		\noindent
		Missionstufe: 1 \\
		Streunendes Monster: Skelett
	}
}
