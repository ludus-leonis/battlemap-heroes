\newcommand{\bmhMissionIIHeadline}{A2: Das Rattenloch}
\newcommand{\bmhMissionIIToc}{A2: Das Rattenloch}
\newcommand{\bmhMissionII}{\fsNormalB\selectfont

	\parchment{
		{\fffancy\noindent
			In der Taverne \say{Zur alten Bastei} machen sich die Gäste über den Wirt lustig. Der traut sich seit Tagen nicht in den eigenen Keller, seit er von Riesenratten fast zu Tode gebissen wurde. Für eine Belohnung von 50\gold~habt ihr euch bereit erklärt, die Rattennester zu suchen und zu zerstören. Ihr steigt die Kellertreppe hinab, während der Wirt hinter euch die Türe verschließt.
		}

		\medskip

		\noindent Die Helden starten auf der Treppe.

		\medskip

		\noindent
		In dieser Mission können Ratten durch Gitter und dank Lüftungsschlitzen in Bodennähe auch durch Türen gehen.

		\trap~Schlagfallen -- es lösen sich Steine von der Decke: -1~LP

		\medskip

		\dropping[2]{1} \say{Der Vorratskeller besitzt Steinwände, die weit älter als das Wirtshaus wirken. Ein Tisch, zwei große Weinfässer an der Westseite und Regale an den Wänden sind die einzigen Möbelstücke hier. Riesige Ratten nagen an Würsten und Getreidesäcken.}

		\search~Die nördlichen Mauern sind zugemauerte Torbögen. Die Ratten haben kleine Tunnel in sie genagt. Dahinter sind weitere Räume oder Gänge auszumachen. Die Mauern haben 5~LP und keine Verteidigung.

		Kommt der SL an die Reihe und sind keine Ratten am Spielplan, aber noch beide Torbögen intakt, erscheint neben jedem eine neue, zusätzliche Ratte.

		\medskip

		\dropping[2]{2} \say{Dieser Gang, der zu mehreren Lagerraumtüren führt, muss zu einer alten Festung gehören. Er wurde wohl seit Jahrzehnten nicht mehr benutzt.}

		\search~In Kisten finden sich jeweils 20\gold.
	}

	\columnbreak

	\vspace*{1cm}

	\newpage

	\vspace*{1cm}

	\columnbreak

	\parchment{
		\dropping[2]{3} \say{Eine Wendeltreppe hat hier früher nach oben geführt, ist aber vor kurzem eingestürzt. Die nördliche Türe ist aus Holz und mit \say{Kerker} beschriftet. Die östliche Türe ist aus Metall und trägt die Aufschrift \say{Rüstkammer}. Die südliche Holztüre ist unbeschriftet, aber mit prunkvollen Schnitzereien verziert.}

		\smallskip

		\dropping[2]{4} \say{Dies muss einmal ein Speisesaal oder Besprechungsraum gewesen sein. In der von Spinnweben verhangenen Südhälfte ist ein alter Kamin auszumachen, über dem zwei gekreuzte Schwerter hängen. In den Netzen hängen zahlreiche tote Ratten. Zwei Riesenspinnen reiben sich die Scheren, als ihr den Raum betretet.}

		\search~Eines der Schwerter ist verrostet und nutzlos, das andere ein Schwert+1.

		\smallskip

		\dropping[2]{5} \say{In der Rüstkammer riecht es nach altem Leder. In einer vergitterten Zelle hängen zahlreiche Rüstungen und Schilde.}

		\search~Der Inhalt wurde Jahrzehnte lang nicht gepflegt und ist leider unbrauchbar geworden.

		\smallskip

		\dropping[2]{6} \say{In dieser Folterkammer mussten vor vielen Jahren Gefangene unvorstellbare Qualen durchmachen.}

		\smallskip

		\dropping[2]{7} \say{Die neun kleinen und großen Zellen werden von Ratten bevölkert. Dies muss ihr Lager sein!}

		\search~In diesen Feldern befinden sich Rattennester. Sie haben 1 LP und keine Verteidigung.

		\smallskip

		\noindent
		Missionstufe: 1 \\
		Streunendes Monster: Ratte
	}
}
