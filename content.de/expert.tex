\newcommand{\bmhExpertChapter}{EXPERTEN-\\REGELN}
\newcommand{\bmhExpertHeadline}{Experten-Regeln}
\newcommand{\bmhExpertToc}{Experten-Regeln}
\newcommand{\bmhExpert}{%

	\dropping{D}ieses Kapitel enthält die Experten-Regeln von \bmh. Ihr könnt jederzeit von den Basis-Regeln umsteigen. Der folgende Text setzt aber nicht voraus, dass ihr diese gelesen habt, sondern wiederholt alle nötigen Regeln.

	\bmhSection{Spielplan}
		Eine \keyword{Battlemap} ist ein roll- oder faltbarer Spielplan, der mit trocken abwischbaren Stiften beschriftet werden kann. Auf diesem Plan ist ein quadratisches Raster aus \keyword[Feld]{Feldern} vorgedruckt.

		Die vier Linien, die jedes Feld umschließen, werden \keyword[Kante]{Kanten} genannt. Die Punkte, an denen vier Kanten zusammenlaufen, heißen \keyword{Ecken}. Kanten und Ecken sind imaginäre Hilfsmittel und nehmen keinen eigenen Platz am Spielfeld ein: Jeder Punkt einer Kante gehört zu beiden Feldern. Ebenso gehört jede Ecke zu allen vier dort zusammenstoßenden Feldern.

		Die umgebenden Felder, mit denen ein Feld eine Kante oder Ecke teilt, sind an dieses \keyword{angrenzend}. Jene, mit denen es nur eine Ecke teilt, werden \keyword[*]{diagonal angrenzend} bezeichnet.

		\example{\bmh\ kann optional auch auf Spielplänen mit wabenförmigen Hex-Feldern gespielt werden. Jedes Feld hat dann nur sechs statt acht angrenzende Felder. Keines davon ist diagonal angrenzend. Alle Regeln behalten ihre Gültigkeit, auch wenn sämtliche Beispiele und Missionen in diesem Band von Quadraten ausgehen. }

	\bmhSection{Rollen \& Missionen}
		\bmh~ist ein Spiel für zwei bis fünf Mitspieler. Einer übernimmt die Rolle des \keyword[Spielleiter]{Spielleiters}\index{SL} (SL), die anderen werden schlicht als \keyword{Spieler} bezeichnet.

		Der SL wählt zu Spielbeginn eine \keyword{Mission} (ab \refPage{lAI}), studiert deren Karte und zeichnet den hervorgehobenen \keyword{Eingangsbereich} auf die Battlemap. Die Spieler wählen inzwischen die Helden, die sie durch die Mission führen wollen (ab \refPage{lHeroCleric}). Ist das deren erste Mission, übertragen die Spieler die Spielwerte auf neue Heldenbögen (\refPage{lSheets}), ansonsten bringen die Helden ihre Werte, Ausrüstung und Erfahrung von vorangegangenen Missionen mit.

		Bei einem, zwei oder vier Spielern besteht die \keyword{Heldengruppe} aus vier Helden, die sich die Spieler untereinander aufteilen. Bei drei Spielern werden allerdings nur drei Helden in die Mission geführt -- dafür erhalten die Helden je einen Heiltrank (+4~LP) mit auf den Weg. Der Trank verfällt nach der Mission, sollte er nicht genutzt wurden.

		Der SL stellt alle seine Kreaturen auf den Spielplan, die sich im Eingangsbereich befinden, dann bittet er die Spieler, ihre Helden auf die im Missionstext erwähnten Startfelder zu stellen. Zu guter Letzt liest der SL den Prolog vor.

		Während der Mission ziehen die Spieler nur ihre Helden, während der SL alle anderen Kreaturen übernimmt.

	\bmhSection{Kreaturen}
		Jede Figur, die einzeln am Spielplan bewegt werden kann, wird als \keyword[Kreatur]{Kreatur} bezeichnet. Meistens sind das Individuen, aber manchmal repräsentiert eine Figur mehrere, kleinere Wesen.

		\example{Der Krieger, der Magier, ein Skelett, ein Riese oder ein Käferschwarm sind Beispiele für Kreaturen.}

		\noindent
		Jedes Feld kann nur von einer Kreatur belegt sein. Diese füllen stets die gesamte, rechteckige \keyword{Grundfläche} ihrer Felder aus, auch wenn die benutze Spielfigur kleiner oder abgerundet sein sollte.

		Die \keyword[Held]{Helden} der Spieler sind auch Kreaturen, wenn auch ganz besondere. Alle Regeln für Kreaturen gelten auch für Helden. Nur wenn explizit von \say{Helden} gesprochen wird, betrifft das nur diese.

		\example{Es kann vorkommen, dass ein Spieler mehrere Kreaturen steuert, oder eine Kreatur den Spieler wechselt, etwa Gefolgsleute oder bezauberte Monster.}

		\noindent
		Alle Kreaturen, die Spieler steuern, gelten als mit einander \keyword{befreundet} -- egal wie lange sie sich bereits kennen. Ebenso sind alle Kreaturen, die der SL steuert, untereinander befreundet. Kreaturen, die nicht miteinander befreundet sind, sind \keyword{befeindet}.

		\example{Von zwei Spielern gesteuerte Krieger und Magier sind mit einander \say{befreundete Kreaturen}. Vom SL gesteuerte Skelette und Riesenratten sind auch mit einander \say{befreundete Kreaturen}. Der Magier und die Riesenratten sind \say{befeindet}.}

	\bmhSection{Attribute \& Proben}
		Die Spielwerte der Kreaturen werden \keyword[Attribut]{Attribute} genannt. Deren gibt es: \keyword{Stärke}\index{ST} (ST), \keyword{Abwehr}\index{AB} (AB), \keyword{Geschick}\index{GE} (GE), \keyword{Reflexe}\index{RE} (RE), \keyword{Intelligenz}\index{IN} (IN) und \keyword{Willenskraft}\index{WI} (WI).

		Ist eine Kreatur aufgefordert, ein Attribut auf die \keyword{Probe} zu stellen, würfelt ihr Spieler entsprechend viele Würfel. Es werden jene gezählt, die gerade Nummern aufweisen. So viele \keyword[Erfolg]{Erfolge} hat die Probe.

		\example{Hat eine Kreatur Stärke~3 und ihr Spieler würfelt 2, 1 und 6. Die 2 und 6 zählen als zwei Erfolge.}

		\noindent
		Sind Proben \keyword{erleichtert}, \zB~\say{GE+2}, oder \keyword{erschwert}, \zB~\say{IN-1}, werden entsprechend mehr oder weniger Würfel benutzt. Sollten nach Abzügen keine Würfel übrig bleiben, hat die Probe automatisch 0 Erfolge.

	\backgroundFooterFig[6.2cm]{%
		\begin{multicols}{2}\raggedbottom
			\color{white}\bmhExpertBoxLOS%
		\end{multicols}
	}

	\bmhSection{Stufen \& Limits}
		Kreaturen und Missionen haben jeweils eine \keyword{Stufe}, die ihre Kompetenz bzw. Gefährlichkeit ausdrückt.

		Da nicht immer alle Helden die selbe Stufe haben, muss vor dem Spiel die passende Missionsstufe bestimmt werden: Dazu wird der gerundete Durchschnitt aller, an der Mission teilnehmenden Helden berechnet. Der SL muss eine Mission wählen, deren Stufe zur Gruppenstufe passt.

		\example{Für Heldengruppe der Stufen 2, 2, 2 und 3 sollte der SL eine Mission der Stufe \say{2} wählen, für eine Gruppe der Stufen 1, 2, 3 und 4 eine Mission der Stufe \say{3}.}

		\noindent
		Während der Mission gilt für alle Kreaturen ein \keyword{Bonus-Limit}: Keine Probe darf um mehr Würfel erleichtert werden, als die Missionsstufe beträgt. Überzählige Würfel verfallen einfach. Alle anderen stufenabhängigen Eigenschaften behalten Kreaturen jedoch und dürfen sie einsetzen, etwa besondere Fähigkeiten.

		\example{Aus Sicht der Helden bedeutet das, dass sich Veteranen aus Rücksicht gegenüber Neulingen etwas zurückhalten und sich Neulinge dafür besonders anstrengen, mit den Veteranen Schritt zu halten.}

	\bmhSection{Runden \& Züge}
		Das Spiel wird in \keyword[Runden]{Runden} abgehalten. Die Mitspieler kommen im Uhrzeigersinn, beginnend links vom SL, an die Reihe und führen für jede ihrer Kreaturen einen Zug aus. Der SL zieht als letzter, danach endet die Runde. Ein Mitspieler muss den Zug einer Kreatur abschließen, ehe mit einer weiteren Kreatur gezogen wird. Ein Mitspieler muss mit allen seinen Kreaturen ziehen -- oder auf deren Zug verzichten -- ehe der nächste Mitspieler an die Reihe kommt.

		Der \keyword[Zug]{Zug} einer Kreatur besteht aus einer Folge von Aktionen, die dessen Spieler bestimmt und durchführt. Die Kreatur erhält dazu so viele \keyword{Aktionspunkte}, wie ihr AP-Wert beträgt.

		Jede \keyword{Aktion} kostet AP. Eine Kreatur muss nicht alle Aktionen zu Beginn des Zuges ansagen, sondern kann den Ausgang einer abwarten, ehe sie weitere setzt. Kreaturen können maximal einen unverbrauchten AP \keyword{ansparen} und in die nächste Runde mitnehmen. Alle anderen AP verfallen am Ende eines Zuges. APs können nicht an andere Kreaturen übertragen werden.

	\bmhSection{Aktionen}
		Kreaturen können folgende Aktionen zu den angegebenen Kosten durchführen:

		\medskip
		\bmhTable{X c}{
			\thead{Aktion} & \thead{Kosten} \\
		}{
			Angreifen           & 2 AP\\
			Benutzen            & 1 AP\\
			Bewegen             & 1 AP\\
			Fähigkeit einsetzen & 0--3 AP \\
			Falle entschärfen   & 2 AP \\
			Rückzug             & 1 AP \\
			Stoßen \& Bugsieren & 1 AP \\
			Suchen              & 2 AP \\
		}
		\medskip

		\bmhSubsection{Angreifen -- 2AP}
			Kreaturen können nur Gegner angreifen, zu denen sie freie oder eingeschränkte Sicht haben (siehe Kasten). Außerdem muss der Gegner in der Reichweite ihrer Waffe stehen: Bestimme die \keyword{Distanz}, indem du die Felder zählst, die du minimal benötigst, um dieses horizontal, vertikal oder diagonal zu erreichen. Ignoriere dabei Hindernisse, Wände oder andere Kreaturen. Das Startfeld zählst du nicht mit, das Zielfeld schon.

			Jede Waffe hat eine \say{vs.}-Angabe. Diese bestimmt, mit welchen Attributen angegriffen und verteidigt wird.

			\example{Eine \say{ST vs. AB}-Waffe führst du mit deiner Stärke, dein Gegner verteidigt mit seiner Abwehr.}

			\noindent
			Hat eine Kreatur nur eingeschränkte Sicht auf ihren Gegner, muss sie mit einem Würfel weniger angreifen. Egal ob Schwert oder Zauberstab, alle Waffen funktionieren nach diesem Prinzip.

			Würfelt der Angreifer mehr Erfolge als der Gegner, hat er getroffen. Für jeden überzähligen Erfolg verliert der Gegner einen Lebenspunkt.

		\bmhSubsection{Benutzen -- 1AP}
			Diese Aktion erlaubt Kreaturen, etwas aus ihrer Ausrüstung oder in ihrem Umfeld zu \keyword{benutzen}. Damit können sie:

			\bmhList{
				\item Dinge auf der Karte benutzen, wenn sie direkt davor stehen, \zB~Türen öffnen/schließen oder Schalter betätigen
				\item einen Trank trinken, den sie bei sich tragen
				\item die Ausrüstung wechseln: Waffe, Rüstung, Schild oder Helm
				\item einer benachbarten Kreatur etwas übergeben
				\item beliebig viele Gegenstände in ihrem oder einem benachbarten Feld aufheben und/oder ablegen
			}

		\backgroundFooterFig[7.4cm]{%
			\begin{multicols}{2}\raggedbottom
				\color{white}\bmhExpertBoxExpose%
			\end{multicols}
		}

		\bmhSubsection{Bewegen -- 1AP}
			Pro Bewegungsaktion darf eine Kreatur bis zu so viele Felder horizontal, vertikal oder diagonal weit \keyword{gehen}, wie ihr BEW-Wert beträgt -- jedes Feld kostet 1 BEW. Dabei darf sie keine Kanten passieren, die von Mauern oder geschlossenen Türen eingenommen werden.

			Am Spielplan mit einem kleinen \say{x} markierte Felder gelten als \keyword{schwieriges Gelände}. Sie zu betreten kostet 1 BEW zusätzlich. Ein Feld, in dem eine andere Kreatur steht, gilt ebenso als schwieriges Gelände. Es zu passieren erfordert ihre Zustimmung.

			Kreaturen können 1 BEW ausgeben, um [liegend] zu werden oder 2 BEW ausgeben, um nicht mehr [liegend] zu sein. Für Kreaturen, die [liegend] sind, zählen alle Felder als schwieriges Gelände. Dafür erhalten sie u.U. Deckung (siehe Kasten \say{Sicht}).

			Zu versuchen, ein Feld nach oben zu \keyword{klettern}, kostet 2 BEW und erfordert eine ST\TN1 Probe. Misslingt die Probe, wird die Bewegung beendet, alle BEW verfallen und die Kreatur fällt. Sie verliert 1 LP pro Feld, dass sie fällt.

			Kann ein Held während deines Zuges \keyword[*]{neue Bereiche einsehen}, wird der SL die Karte vervollständigen (siehe Kasten \say{Aufdecken}).

			Eine Kreatur darf ihre Bewegung nicht auf einem Feld beenden, auf dem bereits eine andere steht. Ungenutzte BEW verfallen am Ende der Aktion.

		\bmhSubsection{Fähigkeit einsetzen -- variabel}
			Die besonderen Fähigkeiten einer Kreatur einzusetzen ist unterschiedlich zeitaufwändig. Deshalb steht in der Beschreibung jeder Fähigkeit, wie viele AP sie kostet. Betragen die Kosten \say{0 AP}, so ist die Fähigkeit permanent aktiv und wird nicht mit dieser Aktion ausgelöst.

			Kreaturen können diese Aktion nur ein mal \emph{pro Missionsstufe} wählen, dürfen aber Fähigkeiten auch mehrfach einsetzen.

			\example{Ein Held der 2. Stufe beherrscht 2 Fähigkeiten und spielt in einer Mission der 3. Stufe mit. Er kann im Laufe der Mission eine der zwei Fähigkeiten 3 Mal einsetzen, oder eine zwei Mal und die andere ein Mal. Später in der Mission trifft der Held auf einen Wiederkehrer. Dieser kann, obwohl er ebenso nur eine Kreatur 2. Stufe ist, seine Fähigkeit -- ein lähmender Schrei -- 3 Mal benutzen.}

			\noindent
			Die Effekte und Auswirkungen sind bei den Helden (ab \refPage{lHeroCleric}) und den Monstern (ab \refPage{lBestiary}) gelistet.

		\bmhSubsection{Falle entschärfen -- 2AP}
			Eine Kreatur kann versuchen, eine Falle in einem angrenzenden Feld zu entschärfen. Sie legt dazu eine Probe ab, die in der Beschreibung der Falle angegeben ist. Gelingt sie, ist die Falle entschärft und wird von der Karte entfernt. Wird bei der Probe kein einziger Erfolg gewürfelt, löst die Falle aus.

		\bmhSubsection{Rückzug -- 1AP}
			Setzt ein Held auf einem der Startfelder die Aktion \keyword{Rückzug}, verlässt er den Spielplan und damit die Mission. Der Held ist dann in Sicherheit und streicht alle Zustände. Er kann nicht wieder in die Mission zurückkehren.

			Haben alle Helden die Mission verlassen, aber das Missionsziel ist noch nicht erfüllt, erhalten die Helden keine Belohnung.

		\bmhSubsection{Stoßen \& Bugsieren -- 1AP}
			Mit dieser Aktion können Kreaturen versuchen, nicht am Zug befindliche Kreaturen zu bewegen. Voraussetzung dafür ist, dass sich die zu bewegende Kreatur in einem angrenzenden Feld und in freier Sicht befindet.

			Der Angreifer hat die Wahl, ob er ST oder GE auf die Probe stellt, der Gegner darf zwischen AB oder RE wählen. Für jeden nicht abgewehrten Erfolg erhält der Angreifer 1 BEW. Diese darf der Angreifer ausgeben, um den Gegner zu den regulären Regeln zu bewegen oder selbst in ein dadurch frei werdendes Feld nachzurücken. Angreifer und Gegner können auch die Plätze tauschen, wenn genügend BEW für beide Bewegungen gleichzeitig vorhanden sind.

		\bmhSubsection{Suchen -- 2AP}
			Ein Held kann jederzeit nach Verborgenem \keyword{suchen}. Dazu legt sein Spieler einen Suchradius fest: ein, zwei oder drei Felder. Durchsucht wird das Feld, auf dem der Held steht, und alle Felder, die bis zu dieser Distanz in Sicht sind. Ob die Suche gelingt, entscheidet eine IN-Probe:

			\bmhTable{X c}{
				\thead{Suchradius} & \thead{nötige Erfolge} \\
			}{
				1 Feld    & 1 \\
				2 Felder  & 2 \\
				3 Felder  & 3 \\
			}

			\noindent
			Werden die nötigen Erfolge erzielt, werden in diesem Radius die Geheimnisse aller mit einem \search~markierten Felder entdeckt. Gefundene Gegenstände dürfen sofort behalten werden, können aber auch liegen gelassen werden, um von anderen später aufgehoben zu werden.

			Manche Felder verbergen \keyword[streunende Kreatur]{streunende Kreaturen}\index{Kreatur!streunend}. Auf jedem sich im Suchradius befindlichen Feld mit einen \monster-Symbol taucht beim erstmaligen Absuchen eine auf. Und zwar egal, ob die IN-Probe erfolgreich war oder nicht! Die Kreaturen werden auf das jeweilige \monster-Feld gestellt (oder möglichst nahe, sollte ein Feld bereits belegt sein). Im Missionstext ist angegeben, welche Kreaturen erscheinen.

	\bmhSection{Gegenangriffe}
		Kreaturen mit einer ST-Waffen bedrohen automatisch alle anderen Kreaturen in deren Reichweite, zu denen sie zumindest eingeschränkte Sicht haben. Löst eine bedrohte Kreatur eine der folgenden Ereignisse aus, erhält jede bedrohende Kreatur die Möglichkeit zum \keyword{Gegenangriff}:

		\bmhTable{X}{
			\thead{Ereignis im bedrohten Feld} \\
		}{
			Eine Kreatur gibt BEW aus. \\
			Eine Kreatur setzt die Aktion \say{Benutzen}. \\
			Eine Kreatur setzt die Aktion \say{Falle entschärfen}. \\
			Eine Kreatur setzt die Aktion \say{Fähigkeit einsetzen}. \\
		}

		\noindent
		Ein Gegenangriff kostet 1 AP und setzt daher voraus, dass die Kreatur einen Aktionspunkt angespart hat. Der Angriff findet statt, bevor die auslösende Kreatur ihre Aktion durchführt oder das Feld verlassen kann. Erst nachdem der Angriff komplett abgehandelt ist, darf die am Zug befindliche Kreatur fortfahren, sollte sie noch dazu in der Lage sein.

	\bmhSection{Zustände}
		Kreaturen können im Spiel temporäre Vor- und Nachteile erlangen. Diese werden \keyword{Zustände} genannt und sind im Text mit Klammern versehen, z.B. \tag{blind}. Eine Kreatur kann einen Zustand nicht mehrfach besitzen. Vor- oder Nachteile unterschiedlicher Zustände zählen gleichzeitig, aber es werden erst positive, danach die negativen verrechnet.

		\example{
			Eine Kreatur kann nicht zwei Mal \tag{schwach} sein, aber \tag{stark} und \tag{schwach}. Verliert diese Kreatur den Zustand \tag{schwach}, so ist sie nur noch \tag{stark}. Hat eine \tag{stark}e Kreatur noch eine weiteren Zustand, die ihr einen Stärke-Bonus gibt, addieren sich diese.
		}

		\bmhTable{l X}{
			\thead{Zustand} & \thead{Auswirkung}\\
		}{
			\tag{agil}         & GE+1, RE+1 \\
			\tag{blind}        & AB-1, RE-1, WI-1, keine Angriffe, kein Aufdecken \\
			\tag{gefangen}     & darf keine BEW ausgeben \\
			\tag{gehemmt}      & ST-1, GE-1, IN-1 \\
			\tag{getarnt}      & bestenfalls eingeschränkte Sicht \\
			\tag{hilflos}      & alle Proben haben 0 Erfolge \\
			\tag{konzentriert} & IN+1 und WI+1 \\
			\tag{langsam}      & BEW-2 \\
			\tag{liegend}      & keine Angriffe möglich \\
			\tag{schnell}      & BEW+2 \\
			\tag{schwach}      & ST-1, AB-1 \\
			\tag{stark}        & ST+1, AB+1 \\
			\tag{träge}        & GE-1, RE-1 \\
			\tag{unsichtbar}   & kann nicht angegriffen werden, kann Feinde passieren \\
		}
		\bmhTable{l X}{
		\thead{Zustand} & \thead{Auswirkung}\\
		}{
			\tag{verflucht}    & verliert einmalig 1 LP am Ende des nächsten Zuges \\
			\tag{vergiftet}    & verliert 1 LP am Ende jedes Zuges \\
			\tag{verwirrt}     & IN-1, WI-1 \\
		}

	\bmhSection{Licht \& Schatten}
		Je nach verfügbarem Licht werden drei Grade an Helligkeit unterschieden:

		\keyword{Hell:} Es ist genügend Licht vorhanden, sodass alle Kreaturen ohne Einschränkungen ihre Aktionen durchführen können.

		\keyword{Düster:} Räume und deren Inhalt sind gerade noch auszumachen und werden vom SL normal eingezeichnet. Kreaturen in düsteren Feldern sind [getarnt] und [gehemmt].

		\keyword{Dunkel:} Es ist Nichts zu sehen. Der SL zeichnet nur Felder ein, die von Helden betreten und damit ertastet wurden. Kreaturen in dunklen Feldern sind [unsichtbar] und [hilflos].

		Jeder Raum bzw. jedes Areal auf dem Spielplan weist eine Grundhelligkeit auf. Ist in der Missionsbeschreibung nichts Gegenteiliges angegeben, so ist sie \say{Dunkel}.

		Kreaturen mit \keyword{Dunkelsicht} dürfen dunkle Felder wie düstere behandeln. Kreaturen mit \keyword{Düstersicht} dürfen düstere Felder wie helle behandeln.

		\bmhSubsection{Lichtquellen}
			Besitzt eine Kreatur eine transportable Lichtquelle, etwa eine Fackel oder Laterne, so erhöht sich die Helligkeit aller Felder in deren Reichweite um einen Grad, zu denen die Kreatur zumindest eingeschränkte Sicht hat. Erleuchten mindestens zwei Lichtquellen ein Feld, ist es \say{hell}.

			Fackeln und Lampen können auf dem Boden abgelegt werden, ohne dass sie erlöschen. Entzündete Fackeln oder Lampen halten bis zum Ende der Mission, müssen dann aber nachgekauft bzw. nachgefüllt werden.

	\bmhSection{Tote Kreaturen}
		Sinken die LP einer Kreatur auf 0, ist sie tot und wird vom Spielplan genommen. Die ehemalige Grundfläche der Kreatur ist nun schwieriges Gelände. Die Ausrüstung und Beute der Kreatur liegt nun ebenfalls auf dem Feld und kann von anderen Kreaturen aufgehoben werden.

	\bmhSection{Riesige Kreaturen}
		Kreaturen, die eine 2×2 Felder große Grundfläche belegen, werden als \keyword{riesig} bezeichnet. Für sie gelten folgende zusätzliche Regeln.

		\keyword[*]{Bewegung:} Riesige Kreaturen bewegen sich, in dem sie für 1 BEW das Zentrum ihrer Grundfläche -- eine Ecke -- auf eine der Ecken am Rand ihrer Grundfläche verschieben. Dadurch nehmen sie zwei, bei diagonaler Bewegung auch drei neue Felder ein. Am Ende ihres Zuges dürfen sie ihre Grundfläche nicht mit anderen Kreaturen teilen. Riesige Kreaturen kostet schwieriges Gelände keine zusätzlichen BEW.

		\keyword[*]{Platzbedarf}: Riesige Kreaturen können nicht durch 1-Feld-große Öffnungen schreiten oder gezwungen werden. Sie können ebenso nicht in derartig kleine Löcher fallen.

		\keyword[*]{Reichweite:} Riesige Kreaturen können die Reichweite jeder Waffe, die sie benutzen, um 1 erweitern.

	\bmhSection[lEquipment]{Geld \& Ausrüstung}
		\bmh~bietet eine Reihe von Gegenständen, die Überleben und Erfolg von Helden sichern können. Sie können in den Missionen gefunden oder zwischen diesen ge- und verkauft werden. Die Währung ist \keyword{G}, \say{tschi} ausgesprochen. Die in diesem Abschnitt angegebenen Preise sind die Kaufpreise. Verkauft wird stets um 75\% des Listpreises (abgerundet). Helden und ihre Freunde dürfen Gegenstände auch untereinander tauschen.

		Einige der Gegenstände müssen in der Hand gehalten werden, um Nutzen zu stiften. Kreaturen haben zumeist zwei Hände. Gegenstände, die zwei Hände benötigen, dürfen mit einer Hand nur getragen aber nicht eingesetzt werden. Möchten Kreaturen ihre Ausrüstung wechseln, müssen sie die Aktion \say{Benutzen} setzen.

		\bmhSubsection{Waffen}
			Jeder Held kann jede Waffe benutzen. Helden dürfen auch mehrere Waffen bei sich tragen, allerdings können sie nur eine davon gleichzeitig einsetzen. Schusswaffen haben stets genügend Munition.

			\bmhTable{X c c c c c}{
				\thead{Waffe} & \thead{Hände} & \thead{vs.} & \thead{RW} & \thead{Preis}\\
			}{
				Armbrust         & 2 & GE vs. AB & 2--8  & 20\gold\\
				Axt              & 1 & ST vs. AB & 1     & 10\gold\\
				Bogen            & 2 & GE vs. AB & 3--10 & 20\gold\\
				Degen            & 1 & GE vs. RE & 1     & 20\gold\\
				Dolch            & 1 & GE vs. AB & 1     & 20\gold\\
				Hammer           & 1 & ST vs. AB & 1     & 20\gold\\
				Hellebarde       & 2 & ST vs. AB & 2     & 20\gold\\
			%	Keule            & 1 & ST vs. AB & 1     &  5\gold\\
				Schleuder        & 2 & GE vs. RE & 2--8  & 10\gold\\
				Schwert          & 1 & ST vs. AB & 1     & 10\gold\\
				Zauberstab       & 1 & IN vs. GE & 2--8  & 50\gold\\
			}

			\noindent
			Besondere Waffen erleichtern ihren Einsatz dank Zusatzwürfel. So gewährt ein \say{Schwert+1 ST~vs.~AB} seinem Nutzer einen zusätzlichen Würfel auf den Angriff.

			\bmhTable{X c c c}{
				\thead{Waffenbonus} & \thead{Preis}\\
			}{
				Waffe +1 & Waffenpreis + 100\gold\\
				Waffe +2 & Waffenpreis + 200\gold\\
				Waffe +3 & Waffenpreis + 300\gold\\
				Waffe +4 & Waffenpreis + 400\gold\\
				Waffe +5 & Waffenpreis + 500\gold\\
			}

		\bmhSubsection{Rüstungen}
			Die folgenden Rüstungsteile schützen ihren Träger, indem sie zusätzliche Würfel auf eine Verteidigung geben.

			\bmhTable{X c c c c}{
				\thead{Rüstung} & \thead{Bonus} & \thead{Malus} & \thead{Preis} \\
			}{
				Helm +1     & AB+1       & --   & 100\gold \\
				Helm +2     & AB+2, RE+1 & IN-1 & 300\gold \\
				Rüstung +1  & AB+1, RE+1 & --   & 150\gold \\
				Rüstung +2  & AB+2, RE+1 & ST-1 & 300\gold \\
				Schild +1   & AB+1       & --   & 100\gold \\
				Schild +2   & AB+2, RE+1 & GE-1 & 300\gold \\
			}

			\noindent
			Von jeder Kategorie (Helm, Rüstung, Schild) darf jede Kreatur nur ein Stück tragen. Schilde erfordern eine freie Hand.

		\bmhSubsection{Ausrüstung}
			Die folgenden Gegenstände können in den Missionen gefunden und zwischen ihnen gekauft werden. Der angegebene Preis ist der Kaufpreis. Verkäufe bringen nur 75\% davon (abrunden).

			\power{Agilitätstrank}{25\gold}{
				Du bist [agil] für 1W6 Runden.
			}

			\power{Fackel}{1\gold}{
				Wird die Fackel angezündet, steigt die Helligkeit in allen Feldern bis RW~4 um einen Grad bis zum Ende der Mission. Danach ist die Fackel nutzlos.
			}

			\power{Gegengift}{50\gold}{
				Du bist nicht mehr [vergiftet].
			}

			\power{Heiltrank}{25\gold}{
				Dieser Trank stellt 2 LP wieder her.
			}

			\power{Konzentrationstrank}{25\gold}{
				Du bist [konzentriert] für 1W6 Runden.
			}

			\power{Lampe}{10\gold}{
				Wird die Lampe angezündet, steigt die Helligkeit in allen Feldern bis RW~6 um einen Grad. Pro Mission verbraucht sie Lampenöl im Wert von 1G.
			}

			\power{Nebelgranate, kleine}{25\gold}{
				Wirf die Granate auf ein Feld, auf das du freie Sicht hast (RW~1--5). Es ist wird bis zum Beginn deiner nächsten Runde in Nebel gehüllt. Es gilt als schwieriges Gelände und blockiert jegliche Sicht.
			}

			\power{Nebelgranate, große}{50\gold}{
				Wirf die Granate auf ein Feld, auf das du freie Sicht hast (RW~1--5). Diese Feld und alle benachbarten Felder, zu denen von dem Feld aus zumindest eingeschränkte Sicht besteht, sind bis zum Beginn deiner nächsten Runde in Nebel gehüllt. Sie gelten als schwieriges Gelände und blockieren jegliche Sicht.
			}

			\power{Schattentrank}{25\gold}{
				Du wirst zum Schatten und kannst bis zum Beginn deiner nächsten Runde Felder durchqueren, in denen ein Feind steht.
			}

			\power{Schnelligkeitstrank}{25\gold}{
				Du bist [schnell] für 1W6 Runden.
			}

			\power{Stärketrank}{25\gold}{
				Du bist [stark] für 1W6 Runden.
			}


			\power{Werkzeug}{50\gold}{
				Du darfst beim Entschärfen 1 Würfel extra benutzen.
			}

			\power{Wünschelrute}{50\gold}{
				Du darfst beim Suchen 1 Würfel extra benutzen.
			}

	\bmhSection[lXP]{Erfahrung}
		Helden, die an einer erfolgreichen Mission teilgenommen haben, erhalten \keyword{Erfahrungspunkte} (EP), und zwar so viele, wie die Stufe der Mission betrug. Haben Helden genügend EP gesammelt, steigen sie eine Stufe auf:

		\bmhTable{c r}{
			\thead{EP} & \thead{Stufe} \\
		}{
			0--4   & 1. Stufe \\
			5--14  & 2. Stufe \\
			15--29 & 3. Stufe \\
			30--49 & 4. Stufe \\
			50--74 & 5. Stufe \\
			ab 75  & 6. Stufe \\
		}

		\noindent
		Auf der 6. Stufe sollte nur mehr eine besondere Abschluss-Mission gespielt werden. Spätestens dann gehen die Helden in Rente und machen einer neuen Generation Platz.

		\bmhSubsection{Aufstieg}
			Erreicht ein Held eine neue Stufe, geschieht folgendes:

			\bmhList{
				\item Der Spieler darf für seinen Helden eine neue Fähigkeit auswählen (siehe Listen der Fähigkeiten der Heldentypen ab \refPage{lHeroCleric}).
				\item Die Missionsstufe muss neu berechnet werden, denn evtl. steigt das Bonus-Limit für alle Helden.
			}

			\noindent
			Es empfiehlt sich, ab der 2. Stufe auf die größeren Heldenbögen zu wechseln.

	\columnbreak
	\bmhSection[lGM]{Spielleiten}

	\bmhSubsection{Fallen}
		Hat ein Feld ein \trap-Symbol, enthält es eine Falle. Die folgenden gewöhnlichen Fallen sind häufiger anzutreffen. Am Ende der Beschreibung in Klammer ist angegeben, mit welcher Probe sie entschärft werden können.

		\keyword{Fallgrube:} Unter einer Klappe befindet sich ein 1 Feld tiefes Loch. Die Kreatur, die eine Fallgrube auslöst, beendet damit ihre Bewegung und fällt automatisch hinein. Die Falle wird danach am Plan eingezeichnet und nimmt das ganze Feld ein. Die fallende Kreatur sowie jede bereits in der Grube befindliche Kreatur verliert 1 LP. (GE\TN1)

		\keyword{Pfeilfalle:} Eine Druckplatte im Boden lässt einen Pfeil aus einer nahen Öffnung schießen und kostet 1 LP. Die Falle kann nur ein Mal auslösen und ist danach automatisch entschärft.

		\keyword{Feuerrune:} Diese Runen-Falle löst aus, indem eine Kreatur sie unabsichtlich verwischt. Sie hat eine Reichweite, die im Missionstext angegeben ist. Eine Explosion richtet dann im Fallenfeld und allen Feldern ihrer Reichweite, zu denen freie Sicht besteht, 1 Feuerschaden an. (IN\TN1)

		Neben diesen gewöhnlichen Fallen gibt es zahlreiche andere Varianten. Enthält eine Mission eine besondere Falle, ist dort nachzulesen, wie diese auslöst und wie schwer sie zu entschärfen ist.

	\bmhSubsection{Eigene Missionen}
		Selbstgeschriebene Missionen sollten folgende Kriterien erfüllen:

		\bmhList{
			\item Jede Mission muss eine Stufe (1, 2, 3\ldots) haben.
			\item Jede Mission muss ihre Stufe und ein streunendes Monster unten rechts erwähnen.
			\item Jede Mission muss einen Prolog oben links enthalten.
			\item Die LP-Summe aller Gegner sollte etwa 15 + Missionstufe*5 betragen.
			\item Missionen müssen mit 3 und 4 Helden lösbar sein.
			\item Jede Mission sollte Schätze oder Ausrüstung im Wert von 200x Missionstufe enthalten. Darunter vorzugsweise eine Rüstung oder Waffe, die einen Bonus in Höhe der Missionsstufe auf ein Attribut gibt.
			\item Riesige Kreaturen dürfen erst ab Missionsstufe 2 auftauchen.
			\item Kreaturen einer Missionsstufe stellen die Mehrheit der normalen Gegner, Kreaturen der Missionsstufe+1 sind ihre Anführer.
		}

	\bmhSubsection{Eigene Kreaturen}
		Neue Monster sollten folgende Eigenschaften haben:

		\bmhList{
			\item Jede Kreatur muss eine Stufe (1, 2, 3\ldots) haben.
			\item Die Grundwerte einer Kreatur mit Stufe X sind: Attribute~X+1; BEW~4, LP~X, AP~3
			\item Davon ausgehend dürfen Zahlen paarweise verschoben werden, z.B. ST um 1 senken und dafür LP um 1 erhöhen.
			\item Die Attribute bewegen sich letztlich zwischen Kreaturstufe (schlecht) und Kreaturstufe+2 (gut).
			\item Für jede besondere Fähigkeit muss ebenfalls eine der Zahlen gesenkt werden. Die Stufe der Kreatur erhöht sich dadurch um 1.
			\item Eine durchschnittliche Bewegung ist BEW 4--5, BEW3 ist langsam, BEW6 schnell.
			\item Durchschnittliche Kreaturen haben soviel LP wie ihre Stufe, zähe Kreaturen 1--2 mehr.
		}

	% \newpage Platzhalter-Seite
}

\newcommand{\bmhExpertBoxLOS}{%
	Um zu bestimmen, ob eine Kreatur \keyword[Sicht]{\color{white}SICHT} zu einem Feld hat, nennt ihr Spieler einen beliebigen Ausgangspunkt in ihrer Grundfläche.

	\hspace{1em}Es besteht \keyword[Sicht, keine]{\color{white}keine Sicht}, wenn von diesem Ausgangspunkt alle Linien zu allen Punkten der Grundfläche des Zieles durch Mauern, geschlossene Türen oder andere, undurchsichtige Hindernisse, wie \zB~dichter Nebel, führt.

	\hspace{1em}Die Sicht ist \keyword[Sicht, freie]{\color{white}frei}, wenn wenn von diesem Ausgangspunkt alle Linien zu allen Punkten der Grundfläche des Zieles weder durch Mauern, geschlossene Türen, undurchsichtige Hindernisse, andere Kreaturen oder maximal die Hälfte des Ziels verdeckendes Mobiliar verlaufen.

	\hspace{1em}Die Sicht ist \keyword[eingeschränkt]{\color{white}eingeschränkt}, wenn weder die Kriterien für freie Sicht noch die für keine Sicht gelten.
}

\newcommand{\bmhExpertBoxExpose}{%
	\keyword[Aufdecken]{\color{white}AUFDECKEN}: Betritt eine Kreatur ein Feld, hat sie die Wahl, welche der Ecken der SL benutzt, um neue Bereiche aufzudecken. Jedes Feld, von dem mindestens die Hälfte von dieser Ecke aus sichtbar ist, wird vollständig eingezeichnet.

	\exampleInverted{Ein Schurke (S) betritt einen Raum, von dem ein weiterer Gang wegführt. Möchte er möglichst viel aufdecken, wird er die nordwestliche Ecke wählen. Möchte er das Risiko auftauchender Gegner minimieren, wird er die südwestliche Ecke wählen.}

	\includegraphics[width=\columnwidth]{\image{bmh/peek-expert.pdf}}
}
