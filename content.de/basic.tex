\newcommand{\bmhBasicChapter}{BASIS-\\REGELN}
\newcommand{\bmhBasicHeadline}{Basis-Regeln}
\newcommand{\bmhBasicToc}{Basis-Regeln}
\newcommand{\bmhBasic}{%

	\dropping{D}ieses Kapitel beschreibt die Basis-Regeln von \bmh. Wenn ihr zum ersten Mal oder mit jungen Helden spielt, solltet ihr euch auf diese Regeln beschränken und zu den erweiterten Regeln des folgenden Kapitels erst übergehen, wenn ihr etwas Übung habt.

	\bmhSection{Kreaturen}
		Jede Spielfigur stellt eine \keyword[Kreatur|basic]{Kreatur} dar. Der Kämpfer, der Magier, ein Skelett -- all das sind Kreaturen. Helden sind auch Kreaturen, nur ganz besondere. Alle Regeln, die für Kreaturen gelten, gelten auch für Helden. Nur wenn explizit von \say{Helden} gesprochen wird, betrifft das nur diese.

	\bmhSection{Proben}
		Immer wieder musst du Attribute auf die \keyword[Probe|basic]{Probe} stellen. Dazu nimmst du so viele Würfel, wie die Zahl des Attributs beträgt, würfelst und zählst jene, die eine gerade Nummer aufweisen. So viele \keyword[Erfolg|basic]{Erfolge} hast du.

		\example{Hat dein Held Stärke~3 und würfelst du 4, 1 und 6, dann zählst du die 4 und 6 als zwei Erfolge.}

		\noindent
		Manchmal wird eine Probe auch leichter oder schwerer. Dadurch ändert sie die Zahl der Würfel, die du wirfst. Solltest du nach Abzügen auf 0 oder weniger Würfel kommen, darfst du nicht würfeln und hast automatisch null Erfolge.

		\example{Beträgt dein Geschick 2 und sollst du \say{GE+2 auf die Probe stellen}, würfelst du 2+2=4 Würfel, bei einer \say{GE Probe} 2+0=2 Würfel und bei einer \say{GE-1 Probe} nur 2-1=1 Würfel.}

	\bmhSection{Runden \& Züge}
		Das Spiel wird in \keyword[Runde|basic]{Runden} abgehalten. Die Spieler kommen, beginnend links vom Spielleiter, im Uhrzeigersinn an die Reihe. Sie führen dann für jede ihrer Kreaturen einen Zug aus. Ein Spieler muss den Zug einer Kreatur abschließen, eher er mit einer weiteren Kreatur zieht und muss mit allen seinen Kreaturen ziehen -- oder auf deren Zug verzichten -- ehe der nächste Spieler an die Reihe kommt. Als Letzter einer Runde zieht der Spielleiter, danach endet die Runde.

	\backgroundFooterFig[7.9cm]{%
		\begin{multicols}{2}\raggedbottom
			\fftext\color{white}\bmhBasicBoxExpose%
		\end{multicols}
	}

		Der \keyword[Zug|basic]{Zug} einer Kreatur besteht aus einer Folge von Aktionen, die deren Spieler bestimmt und durchführt. Der Spieler hat dabei die Wahl, ob seine Kreatur~\ldots

		\bmhList{
			\item sich bewegt oder etwas benutzt, dann
			\item angreift oder sucht
		}

		{\centering\ldots~oder~\ldots}

		\bmhList{
			\item angreift oder sucht, dann
			\item sich bewegt oder etwas benutzt
		}

		{\centering\ldots~oder~\ldots}

		\bmhList{
			\item eine Fähigkeit einsetzt
		}

		{\centering\ldots~oder~\ldots}

		\bmhList{
			\item sich bewegt oder etwas benutzt, dann
			\item sich nochmals bewegt oder etwas benutzt, dann
			\item sich nochmals bewegt oder etwas benutzt.
		}

		\noindent
		Ein Spieler muss nicht alle Aktionen zu Beginn des Zuges ansagen, sondern kann den Ausgang einer Aktion abwarten, ehe er weitere Aktionen plant.

	\backgroundFooterFig[6.9cm]{%
		\begin{multicols}{2}\raggedbottom
			\fftext\color{white}\bmhBasicBoxDistance%
		\end{multicols}
	}

	\bmhSection{Aktionen}

		\noindent
		Dieser Abschnitt erklärt dir, was dein Held mit den \keyword[Aktion|basic]{Aktionen} machen kann.

		\bmhSubsection{Bewegen}
			Mit einer Bewegung darf eine Kreatur bis zu so viele Felder weit gehen, wie ihr Wert in BEW aussagt. Kreaturen bewegen sich dabei von Feld zu Feld und dürfen natürlich nicht durch geschlossene Türen oder Mauern gehen. Eine Kreatur darf ihre Bewegung nicht auf einem Feld beenden, auf dem bereits eine andere steht.

			Sich ein Feld horizontal oder vertikal zu bewegen kostet 1 BEW. Sich diagonal zu bewegen kostet ebenfalls 1 BEW, ist aber nur dann erlaubt, wenn das Zielfeld über das horizontal oder vertikal dazwischen liegende Feld erreichbar wäre.

			Gelangst du während des Zuges neben ein unbekanntes Feld, kannst oder musst du den Spielleiter bitten, aufzudecken (siehe Kasten). Danach kannst du dich weiter bewegen oder deinen Zug gemäß den neuen Erkenntnissen anpassen.

			Manche Felder werden durch Geröll, Treppen, Glatteis oder andere Hindernisse zu \keyword[Gelände!schwierig|basic]{schwierigem Gelände}. Sie werden vom Spielleiter durch ein kleines \say{x} am Spielplan markiert und kosten 1 BEW zusätzlich. Ein Feld, in dem eine andere Kreatur steht, gilt ebenso als schwieriges Gelände. Es zu durchschreiten ist allerdings nur mit ihrer Zustimmung erlaubt.

			Du kannst während deiner Bewegung \keyword[Tür|basic]{Türen öffnen oder schließen}. Dazu musst du direkt vor der Türe stehen und 1 BEW ausgeben. Der Spielleiter markiert sie dann entsprechend auf der Karte.

			Gegenstände aus deinem oder den acht benachbarten Feldern \keyword[Gegenstand!aufheben/ablegen|basic]{aufheben oder ablegen} kostet 1 BEW pro Gegenstand.

		\bmhSubsection{Benutzen}
			Mit dieser Aktion kann dein Held etwas aus seiner Ausrüstung \keyword{benutzen}, etwa einen \keyword[Trank!trinken|basic]{Trank trinken}\index{trinken|basic} oder die \keyword[Waffe!wechseln|basic]{Waffe wechseln}\index{wechseln (Waffe)|basic}.

			Dein Held kann mit dieser Aktion aber auch mit Gegenständen \keyword[interagieren|basic]{interagieren}, etwa Schalter betätigen oder Kisten verschieben -- im Missionstext steht, was dann passiert. Türen zu öffnen ist jedoch Teil einer Bewegungsaktion.

		\bmhSubsection{Angreifen}
			Nur wenn du Sicht zum Gegner hast und er in der Reichweite deiner Waffe steht, kannst du ihn angreifen.

			Ziehe eine Linie vom Mittelpunkt deines Feldes zum Mittelpunkt des Feldes des Gegners. Wird diese Linie durch eine Wand oder ein Hindernis blockiert, hast du \keyword[Sicht!keine|basic]{keine Sicht}. Streift die Linie bloß eine Ecke, oder durchquert sie ungehindert von Mauern nur die Felder anderer Kreaturen, hast du \keyword[Sicht!eingeschränkt|basic]{eingeschränkte Sicht}. Kannst du eine komplett ungehinderte Linie ziehen, hast du \keyword[Sicht!freie|basic]{freie Sicht}.

			Bestimme danach gemäß dem Kasten \say{Distanz} die Entfernung zu deinem Gegner. Nur wenn deine Waffe eine passende \keyword[Reichweite|basic]{Reichweite} (RW) hat, kannst du den Angriff durchführen.

			Sieh jetzt bei deiner Waffe nach, was ihre \keyword[vs.|basic]{vs.-Angabe} ist. Das bestimmt, mit welchen Proben angegriffen und verteidigt wird.

			\example{Eine \say{ST vs. AB}-Waffe führst du mit deiner Stärke, dein Gegner verteidigt mit seiner Abwehr.}

			\noindent
			Hast du nur eingeschränkte Sicht auf deinen Gegner, darf er sich mit einem Würfel zusätzlich verteidigen. Egal ob Schwert oder Zauberstab, alle Waffen funktionieren nach diesem Prinzip.

			Mach nun eine Probe auf das Angriffs-Attribut und gib dem Gegner die Gelegenheit, eine Probe auf das Verteidi\-gungs-Attribut zu machen. Für jeden Erfolg, den der Angriff mehr als die Verteidigung hat, verliert der Angegriffene einen \keyword[Lebenspunkte|basic]{Lebenspunkt}. Kreaturen, deren Lebenspunkte auf 0 fallen, sind \keyword[tot|basic]{tot} und werden vom Spielplan genommen. Bei Gegnern mit vielen Lebenspunkten führt ihr am Rand der Battlemap Buch, wie viele diese noch besitzen.

		\bmhSubsection{Fähigkeiten einsetzen}
			Jeder Held darf ein Mal pro Mission seine \keyword{Fähigkeit!einsetzen|basic}{besondere Fähigkeit einsetzen}. Dies nimmt einen ganzen Zug in Anspruch.

		\bmhSubsection{Suchen}
			Sind keine Gegner auf dem Spielplan, kann dein Held die Zeit nutzen, um nach Verborgenem zu \keyword[suchen|basic]{suchen}. Mach eine IN-Probe. Hast du mindestens einen Erfolg, muss dir der Spielleiter alle Schätze, Fallen oder Geheimnisse nennen, die sich auf deinem Feld und jenen der den acht direkt angrenzenden Feldern befinden, zu denen du freie Sicht hast. Schätze musst du aber erst aufheben (siehe \say{Bewegung}).

			Suchen birgt aber das Risiko, dass eine \keyword[streunende Kreatur|basic]{streunende Kreatur}\index{Kreatur!streunend} auf dich aufmerksam wird. Gelingt die IN-Probe nicht, hast nicht nur nichts gefunden, sondern du hast auch eine Kreatur aufgeschreckt. In jeder Mission ist angegeben, welche das ist. Der Spielleiter stellt sie auf ein dir benachbartes Feld (oder ein möglichst nahes, sollten alle Felder belegt sein).

			Tipp: Ihr könnt erfolgreich durchsuchte Felder direkt am Spielplan markieren.
}

\newcommand{\bmhBasicGMHeadline}{Spielleiten}
\newcommand{\bmhBasicGMToc}{Spielleiten}
\newcommand{\bmhBasicGM}{%

	\bmhSection{Spielleiten}
	Als \keyword[Spielleiter|basic]{Spielleiter}\index{SL|basic} (SL) steuerst du keine Helden, sondern all die Kreaturen, die den Helden entgegen treten. Das ist weniger kompliziert und macht viel mehr Spaß, als es sich liest!

	Außerdem ist das kartographieren des Gemäuers deine wichtigste Tätigkeit. Achte bei jedem Schritt eines Helden darauf, ob du gemäß der Aufdecken-Regel neue Areale einzeichnen musst. Aber nicht alle Informationen der Mission sind für die Helden sofort sichtbar. Du kannst bedenkenlos alle Felder aufzeichnen, die kein Symbol in der unteren linken Ecke haben. Falls da aber doch ein Symbol ist, musst du aufpassen und den dazugehörigen Text lesen. Da diese Symbole für den Spielleiter bestimmt sind, werden sie natürlich nicht auf der Battlemap eingezeichnet.

	\bmhList{
		\item[\trap] Fallen werden am Plan nicht eingezeichnet.

		\item[\search] Diese Felder haben eine Besonderheit. Zeichne das Feld vollständig ein, aber lasse die Spieler nicht wissen, dass das Feld etwas verbirgt, das nicht im Vorlesetext erwähnt wurde. Wenn ein Held hier sucht, erfährt er die zusätzlichen Informationen aus dem Missionstext.
	}

	Nachdem du das Spiel wie in \say{\bmhMiscPreparationHeadline} angegeben mit dem Vorlesen des Prologs und dem Aufzeichnen des punktierten Eingangsbereichs begonnen hast, kannst du dich erst einmal zurück lehnen und die Spieler handeln lassen.

	Sollten Monster auf dem Spielplan sein, darfst du mit diesen ziehen, wenn du an die Reihe kommst. Du darfst aber nur Monster ziehen, die bereits aufgedeckt wurden. Sie haben die selben Möglichkeiten wie die Helden, können aber nicht Suchen oder Aufdecken. Sie sind mit allen Fallen vertraut und können daher Fallenfelder durchschreiten, ohne diese auszulösen.

	\bmhSubsection{Fallen}
		Felder mit einem \trap-Symbol enthalten \keyword[Falle|basic]{Fallen}. Solange ihr mit den Basis-Regeln spielt, ignoriere die Beschreibung der Falle im Missionstext -- alle Fallen sind dann~\ldots

		\keyword[Falle!Pfeil-|basic]{Pfeilfallen}: Betritt ein Held ein Fallenfeld, ohne es vorher erfolgreich durchsucht zu haben, löst eine Falle aus: Eine Platte im Boden macht \say{Klick} und ein Pfeil kommt angeflogen. Der Held verliert 1~LP, und seine Bewegung endet. Die Falle ist damit aber auch entschärft. Durchsucht ein Held erfolgreich ein Fallenfeld, wird die Falle entdeckt und automatisch entschärft.

	\bmhSubsection{Missionen}
		Du kannst mit den Basis-Regeln alle \keyword[Mission|basic]{Missionen} spielen, die für Helden der 1. Stufe geschrieben sind. Ab der 2. Stufe solltest du jedoch auf die Experten-Regeln umsteigen.
}

\newcommand{\bmhBasicBoxExpose}{%
	\keyword[aufdecken|basic]{AUFDECKEN}: Gelangt ein Held -- \zB~der Krieger (K) -- auf ein diagonal an Unbekanntem angrenzendes Feld \emph{kann} sein Spieler den SL bitten, aufzudecken. Gelangt der Held -- \zB~ der Schütze (S) -- auf ein Feld, das eine Seite mit Unbekanntem teilt, \emph{muss} der SL sofort aufdecken. Der Magier (M) ist nicht angrenzend und darf \emph{nicht} aufdecken lassen.

	\hspace{1em}Wird aufgedeckt, muss der Spielleiter den angrenzenden Gang oder Raum komplett einzeichnen, etwaige Monster auf den Spielplan stellen und weiteren Beschreibungstext vorlesen, falls das neue Areal einen hat.

	\medskip
	\includegraphics[width=\columnwidth]{\image{bmh/peek-basic.pdf}}
}

\newcommand{\bmhBasicBoxDistance}{%
	Um die \keyword[Distanz|basic]{DISTANZ} von einem Feld zu einem anderen zu ermitteln, zähle die Felder, die du minimal benötigst, um dieses auf direkter Linie horizontal, veritkal oder diagonal zu erreichen. Ignoriere dabei Hindernisse, Wände oder andere Figuren. Das Startfeld zählst du nicht mit, das Zielfeld schon.

	\medskip

	\emph{Die Goblins (G) sind vom Krieger (K) ein Feld entfernt, die Orks (O) zwei und das Skelett (S) vier.}

	\centering\includegraphics[width=\columnwidth]{\image{bmh/distance.pdf}}
}
