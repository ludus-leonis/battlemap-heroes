\newcommand{\bmhChangelogHeadline}{Letzte Änderungen}
\newcommand{\bmhChangelogToc}{Letzte Änderungen}
\newcommand{\bmhChangelog}{%

	\dropping{D}ie folgenden Änderungen wurden am Regelwerk gemacht. Ein \say{B} weist auf Basis-Regeln und \say{E} auf Experten-Regeln hin. Ein \say{M} steht für Mission und ein \say{A} für sonstige, allgemeine Änderungen. \say{L} steht für Layout.

	\bmhList {\small
		\item     \textbf{v0.1 -- Alpha}
		\item[B]  erster Draft
		\item     \textbf{v0.2 -- Alpha}
		\item[I]  dieses ChangeLog eingeführt
		\item[E]  Experten-Regeln (Entwurf)
		\item[E]  Aktionspunkte
		\item[B]  SL-Anleitung erweitert
		\item[BE] Fallen
		\item[BE] Sicht
		\item[E]  Stufen für Helden und Gegenstände
		\item[BE] Besondere Fähigkeiten
		\item[L]  Layout und Symbole verbessert
		\item[E]  Waffen und Rüstungen (Entwurf)
		\item[E]  Bestiarum
		\item[M]  A1 überarbeitet
		\item     \textbf{v0.3 -- Alpha}
		\item[M]  neue Mission: Das Rattenloch
		\item[E]  Erfahrungspunkte und Aufstieg
		\item[E]  (Gruppen)Stufen und Bonus-Limit
		\item[A]  Abenteuer->Mission, Figur->Kreatur
		\item[BE] Aktion: Benutzen
		\item[A]  Gegenstände-Kapitel (Entwurf)
		\item[BE] Waffenschaden gemäß Erfolgen
		\item[A]  Bestiarium: Goblin, Ork, Spinne, Ratte, Leitfaden für eigene Monster
		\item[M]  A1 Wasserbecken angepasst
		\item[E]  Tipps zum Schreiben von Missionen (Entwurf)
		\item[A]  größere Heldenbögen für erfahrene Helden
		\item[L]  Beispiel-Boxen, Tabellen, Bestiarium-Icons
		\item[A]  Helden-/Monsterwerte angepasst
		\item     \textbf{v0.4 -- Alpha}
		\item[E]  Gegen-/Gelegenheitsangriffe
		\item[E]  Aktion: Stoßen und Bugsieren
		\item[E]  Licht und Lichtquellen
		\item[M]  optisch hervorgehobene Eingangsbereiche
		\item     \textbf{v0.5 -- Alpha}
		\item[E]  Zustände
		\item[E]  mehr Tränke
		\item[L]  Seitenzahlen in der Kopfleiste
		\item[A]  Index
		\item[E]  mehr Erklärung zu Feldern, Missionen, Rollen und Attributen
		\item[E]  große Kreaturen
		\item[E]  Begriff \say{Grundfläche} eingeführt
		\item[M]  aufgehübschte Karten
		\item[E]  Sicht-Bestimmung präzisiert
		\item     \textbf{v0.6 -- Beta}
		\item[E]  7 exklusive Fertigkeiten für jeden Helden
		\item[E]  Zustände für ungeügende Lichtverhältnisse
		\item[E]  Zustand [liegend]
		\item[E]  Aktion: Rückzug
		\item[E]  Tod von Kreaturen
		\item[E]  Hilfestellung zu eigenen Missionen/Monstern aktualisiert
		\item[M]  aufgehübschte Missionsseiten
		\item[A]  Kapiteldeckblätter
		\item[A]  Abschaltbare (schwarze) Hintergründe für den Druck
		\item[A]  Schatz- und Infosymbol durch Suchen-Symbol ersetzt
		\item[A]  SL-Schirm als nötiges Spielmaterial gelistet
		\item     \textbf{v0.7 -- Beta}
		\item[BE] \say{benutzen} vereinheitlicht
		\item[BE] \say{diagonal bewegen} vereinfacht
		\item[E]  Klarstellung und Beispiel \say{Fähigkeit einsetzen}
		\item[A]  Magier: Schutzzauber geändert
		\item[M]  Skelette in A1 umverteilt
		\item[A]  kleines Lektorat
		\item[E]  Ausrüstungsabschnitte zusammengezogen
		\item[A]  PDF-Inhaltsverzeichnis korrigiert
		\item     \textbf{v0.8 -- Beta}
		\item[M]  A2 leicht überarbeitet
		\item[M]  Fallen/Suchsymbole in Kreise gesetzt
		\item[A]  neue Monstersymbole
		\item[L]  Spalten-Layout im Bestiarium
		\item[BE] Suchen und streunende Monster abgeändert
		\item[BE] alle Helden starten mit +1 LP
		\item[BE] Regeln für 2, 3 und 4 Spieler

	}

	% \newpage \vspace*{1cm}
}
